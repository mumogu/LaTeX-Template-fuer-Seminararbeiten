% Kurzformen für übliche Zahlenmengen 
\newcommand{\N}{\ensuremath{\mathbb{N}}}
\newcommand{\Z}{\ensuremath{\mathbb{Z}}}
\newcommand{\Q}{\ensuremath{\mathbb{Q}}}
\newcommand{\R}{\ensuremath{\mathbb{R}}}
\newcommand{\C}{\ensuremath{\mathbb{C}}}

% Weitere Kurzformen für Mathematikumgebungen
\newcommand{\abs}[1]{\ensuremath{\left\vert#1\right\vert}}
\newcommand{\limes}[2]{\ensuremath{\underset{#1}{\lim} #2}}
\newcommand{\limesni}[1]{\ensuremath{\underset{n \to \infty}{\lim} #1}}
\newcommand{\qed}{\hfill \ensuremath{\Box}}
\newcommand{\p}{\ |\ }
\newcommand{\ra}{\ensuremath{\rightarrow}}
\newcommand{\Ra}{\ensuremath{\Rightarrow}}
\newcommand{\infinity}{\ensuremath{\infty}}
\renewcommand{\i}{\text{i}}
\newcommand{\textfrc}[1]{{\frcseries#1}}
\newcommand{\mathfrc}[1]{\text{\textfrc{#1}}}
\newcommand{\ueb}{\ensuremath{{}_{\scriptscriptstyle 1}\!}}
\newcommand{\ph}{\ensuremath{\phantom{\ueb}}}

% Schriftart für Mathematikumgebungen
\DeclareMathAlphabet{\mathpzc}{OT1}{pzc}{m}{it}