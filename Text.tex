% Document Class muss zuerst definiert werden.
\documentclass[12pt, a4paper]{scrartcl}

\usepackage{fancyhdr}
\usepackage[usenames,dvipsnames]{xcolor}
\usepackage[ngerman]{babel}
\usepackage{geometry}
\usepackage{graphicx}
\usepackage{wrapfig}
\usepackage{setspace}
\usepackage[utf8]{inputenc}
\usepackage{amsmath}
\usepackage{amssymb}
\usepackage{pgf}
\usepackage{tikz}
\usepackage{listings}
\usepackage{parcolumns}
\usepackage{enumerate}
\usepackage{color}
\usepackage{mathabx}
\usepackage{mathdots}
\usepackage{dsfont}
\usepackage{ulem}
\usepackage{listings}
\usepackage{microtype}		% Für hängende Punktuation

% Verhindert, dass der untere Rand der Seite mit Whitespace ausgeglichen wird.
\raggedbottom

% Zeilenabstand einstellen
\onehalfspacing

% Eigenen Seitenstil aktivieren
\pagestyle{fancy} 

% Alle Kopf- und Fußzeilenfelder bereinigen
\fancyhf{} 

% Obere Trennlinie
\renewcommand{\headrulewidth}{0.7pt} 

% Seitennummer unten in der Mitte 
\fancyfoot[C]{\thepage} 

\geometry{a4paper, portrait, left=2cm, right=2cm, top=2cm, bottom=2cm, head=2cm, foot=1cm}

\fancyhead[L]{}
\fancyhead[C]{}
\fancyhead[R]{}

\begin{document}
\section*{Zusammenfassung von:\\Rudolf Carnap: „Scheinprobleme in der Philosophie"\footnote{Zitiert nach Hubert Schleichert, \textit{Von Platon bis Wittgenstein. Ein philosophisches Lesebuch}. München: Beck, 1998, S. 22-29}}
Rudolf Carnap behandelt in seinem Text „Scheinprobleme in der Philosophie“ die Frage wann eine Aussage sinnvoll ist und wann eine wissenschaftliche Auseinandersetzung mit einer Aussage oder These gerechtfertigt ist. 

Ein grundlegender Begriff in Carnaps Argumentation ist die Sinnhaftigkeit einer Aussage. Er unterteilt alle Aussagen in solche die sinnvoll sind und solche die sinnlos sind. Damit eine Aussage nach Carnaps Definition sinnvoll ist, muss sie einen Sachverhalt beschreiben. Es muss möglich sein ein Experiment anzugeben, an dessen Ausgang man beobachten kann, dass die Aussage wahr ist. Sie muss also nachprüfbar sein.\footnote{Carnap, S. 22} Ob dieses Experiment praktisch durchführbar ist spielt für die Sinnhaftigkeit einer Aussage keine Rolle. 

Dieser Begriff der Sinnhaftigkeit ist insbesondere unabhängig vom konkreten Wahrheitswert der Aussage. Es lässt sich für jede Aussage bestimmen ob sie sinnvoll ist oder nicht, auch wenn ihr tatsächlicher Wahrheitswert unbekannt ist. 

Carnaps Definition folgend, sind die beiden Aussagen „dieser Stein ist traurig“ und „bu ba bi“ nicht sinnvoll\footnote{Carnap, S. 23}. Es lässt sich nämlich kein Experiment angeben, mit dem sich nachprüfen lässt, dass ein Stein traurig ist. Des Weiteren unterscheiden sie sich nicht systematisch. Beide Aussagen sind gleichermaßen sinnlos und eignen sich nach Carnaps Meinung nicht für eine Wissenschaftliche Betrachtung.

Aussagen die sich heute nicht mehr durch ein Experiment verifizieren lassen, wie sie beispielsweise in den Geschichtswissenschaften vorkommen, bezeichnet Carnaps ebenfalls als sinnvoll.\footnote{Carnap, S. 24} Auch hier muss sich allerdings ein konkretes Erlebnis angeben lassen, an dem sich der Wahrheitswert der Aussage ablesen lässt. 

Carnap behauptet, dass sich alle Realwissenschafften ausschließlich mit sinnvollen Aussagen beschäftigen und die Philosophie und die Theologie die einzigen Wissenschaften sind, die sich auch mit sinnlosen Aussagen auseinandersetzen.

Carnap fasst in weiteren Verlauf des Textes die grundlegenden Thesen des Realismus und des Idealismus wie folgt zusammen. Der Realismus besagt, dass alle Gegenstände, die eine Person wahrnimmt auch außerhalb ihrer Wahrnehmung, also in der realen Welt existieren, und des Weiteren andere Menschen und ihre Wahrnehmungen auch in der realen Welt existieren. Der Idealismus hingegen besagt, dass die Außenwelt nur in meiner Wahrnehmung existiert, und demnach alle Bewusstseinsvorgänge anderer Menschen reine Konstruktionen meines eigenen Bewusstseins sind. 

Um den Unterschied zwischen Realismus und Idealismus zu verdeutlichen, führt Carnap ein Gedankenexperiment durch. Es begeben sich zwei Geologen, einer davon Realist, der andere Idealist, auf die Suche nach einem mystischen Berg. Sie machen sich auf die Suche, finden den Berg und werden sich einig darüber, dass er existiert und über alle seine Eigenschaften in der empirischen Welt wie Höhe oder seine geologische Beschaffenheit. Der Unterschied zwischen den beiden zeigt sich erst in der Deutung der Ergebnisse. Von einem realistischen Standpunkt aus betrachtet zeugt die empirische Erfahrung von der Tatsächlichen Existenz des Berges. Der Idealist deutet die Erfahrung hingegen als Indiz, dass der Berg nur in seiner eigenen Wahrnehmung real ist.\footnote{Carnap, S. 27f}

Carnap beizieht keine Stellung zur Frage ob der Realismus oder der Idealismus hier näher an der Wirklichkeit liegen. Vielmehr wendet er seinen zuvor eingeführten Begriff der Sinnhaftigkeit auf die Thesen der beiden Theorien an. Er stellt fest, dass beide Theorien keine realen Sachverhalte beschreiben, sich also keine konkrete Wahrnehmung angeben lässt, die eine der beiden Theorien verifiziert oder falsifiziert. Demnach handelt es sich nach Carnaps Definition um reine Scheinaussagen und keine Wissenschaft wird jemals in der Lage sein zu entscheiden, welche der beiden Theorien stichhaltiger ist.

\end{document}